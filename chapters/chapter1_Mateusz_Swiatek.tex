\section{Mateusz Świątek\item \item }

Memik na dobry początek dnia (Figure~\ref{fig:altF4})

\begin{figure}[h] % kontrolowanie gdzie będzie dany element h-tutaj t-top b-bot p-next page (top) 
    \centering
    \includegraphics[width=0.7\textwidth]{pictures/altF4.jpg} %trzeba zwiększyć współczynnik 0.2
    \caption{Tak będzie, nie zmyślam}
    \label{fig:altF4} %skrót do tego elementu
\end{figure}

% Dlaczego wyrażenie przeskoczyło do nowej linijki? bo jest \ co symbolizuje przejście do nowej linijki
\newpage


Zoba jakie fajny wzorek(Eulera): \[e^i^\pi + 1 = 0\]

a tutaj wzór na kółko
$ x^2 + y^2 = 1 $\\

W Tabeli~\ref{tab:Sudoku1} posiadasz zadanko na dziś.\\ % Do czego służy \ref{}? do odwołań do poszczególnych elementów w tym pliku


\begin{table}[h]
\centering
\begin{tabular}{|c|c|c|c|c|c|c|c|c|} % c-central r-right l-left | robi linię pomiędzy kolumnami \hline-pozioma linia
\hline
\textbf{5} & \textbf{}  & \textbf{}  & \textbf{}  & \textbf{7} & \textbf{}  & \textbf{}  & \textbf{}  & \textbf{}  \\ \hline
\textbf{}  & \textbf{3} & \textbf{}  & \textbf{}  & \textbf{}  & \textbf{9} & \textbf{}  & \textbf{2} & \textbf{8} \\ \hline
\textbf{}  & \textbf{}  & \textbf{}  & \textbf{}  & \textbf{6} & \textbf{3} & \textbf{7} & \textbf{}  & \textbf{1} \\ \hline
\textbf{}  & \textbf{}  & \textbf{}  & \textbf{}  & \textbf{}  & \textbf{8} & \textbf{2} & \textbf{}  & \textbf{}  \\ \hline
\textbf{}  & \textbf{7} & \textbf{2} & \textbf{}  & \textbf{9} & \textbf{}  & \textbf{}  & \textbf{}  & \textbf{}  \\ \hline
\textbf{9} & \textbf{}  & \textbf{5} & \textbf{}  & \textbf{}  & \textbf{}  & \textbf{}  & \textbf{7} & \textbf{6} \\ \hline
\textbf{6} & \textbf{}  & \textbf{}  & \textbf{1} & \textbf{8} & \textbf{}  & \textbf{}  & \textbf{}  & \textbf{}  \\ \hline
\textbf{}  & \textbf{1} & \textbf{7} & \textbf{9} & \textbf{3} & \textbf{}  & \textbf{6} & \textbf{8} & \textbf{4} \\ \hline
\textbf{8} & \textbf{5} & \textbf{3} & \textbf{2} & \textbf{4} & \textbf{}  & \textbf{9} & \textbf{}  & \textbf{7} \\ \hline
\end{tabular}
\caption{Takie tam zadanko - poziom EASY}
\label{tab:Sudoku1}
\end{table}\\
Wydziały na AGH
\begin{itemize}
  \item [\$] Wydział Elektrotechniki, Automatyki, Informatyki i Inżynierii Biomedycznej \verb|(Najlepszy)|
  \item [?]Wydział Inżynierii Metali i Informatyki Przemysłowej
  \item [WIET] Wydział Informatyki, Elektroniki i Telekomunikacji
  \begin{description}
     \item[Note:] Zagrożenie życia lub zdrowia
    \end{description}
  \item [\<>]Wydział Inżynierii Lądowej i Gospodarki Zasobami
  \item itd \\
\end{itemize}

Oraz najtrudniejsze przedmioty:
\begin{enumerate}
  \item Analiza \texttt{Matematyczna}
  \item Algebra Liniowa i Geometria analityczna 
  \item Podstawy Programowania
  \item WF
\end{enumerate}
\newpage

\begin{center}
\resizebox{!}{0,85cm}{Hobbit}\\
\end{center}

{\huge W} pewnej norze ziemnej mieszkał sobie pewien \textbf{hobbit} . Nie była to szkaradna, brudna, wilgotna nora, rojąca się od robaków i cuchnąca błotem, ani też sucha, naga, piaszczysta nora bez stołka, na którym by można usiąść, i bez dobrze zaopatrzonej spiżarni; była to nora hobbita, to znaczy: nora z wygodami.

Miała drzwi \underline{doskonale} okrągłe jak okienko okrętowe, pomalowane na \textbf{\textit{zielono}},
z lśniącą, żółtą mosiężną klamką, sterczącą dokładnie pośrodku. Drzwi prowadziły do hallu, który miał kształt rury i wyglądał jak \emph{tunel}: był to bardzo wygodny
tunel, nie zadymiony, z boazerią na ścianach i chodnikiem na kafelkowej podłodze; nie brakowało tu politurowanych krzeseł ani mnóstwa wieszaków na kapelusze i płaszcze, bo hobbit bardzo lubił gości. Tunel wił się w skrętach, wił się i wił,
wdrążając się głęboko, choć wcale nie prostą drogą, we wnętrze pagórka — a raczej: Pagórka, bo tak go nazywano w promieniu wielu mil — a mnóstwo okrągłych drzwiczek otwierało się to po jednej, to po drugiej jego stronie. Hobbici nie
uznają schodów. Sypialnie, łazienki, piwnice, spiżarnie (mnóstwo spiżarni!), garderoby (hobbit miał kilka pokoi przeznaczonych wyłącznie na ubrania), kuchnie,
jadalnie — wszystko mieściło się na tym samym piętrze, a nawet wzdłuż tego samego korytarza. Najparadniejsze pokoje znajdowały się z lewej strony, (patrząc
od wejścia), ponieważ tylko te miały okna, głęboko osadzone, okrągłe okna z widokiem na ogród, a dalej na łąki zbiegające w dół ku rzece. 

\vspace{2cm}
Bilbo razem z Gandalfem zrobią za nas projekt, dalsza historia nastapi za rok: 