\section{Gabriela Bułat\item}
Na dole tez się znajdzie tabelka~~\ref{moje}\\

\begin{figure}
    \centering
    \includegraphics[width= 5cm, height=6cm]{pictures/kot.jpg}
    \label{fig:my_label}
\end{figure}\\

\newpage
Dlaczego koty są cudowne: (obrazek~\ref{fig:my_label})
\begin{itemize}
    \item są super mięciutkie
    \item dają najlepsze przytulasy
    \item są pocieszne i dają życiu radości
    \item stanowią świetne towarzystwo w zimne wieczory\\
\end{itemize}

Ukochane jedzonko:
\begin{enumerate}
    \item Pierogi (ruskie w szczególności)
    \item Spaghetti
    \item Kopytka\\
\end{enumerate}

Najbardziej kojarzący się z liceum wzór -  \(a^2 + b^2 = c^2\)\\
\newpage
\begin{center}
    \resizebox{!}{0,5cm}{Quite random Harry Potter}
\end{center}

\textbf{Harry Potter i Kamień Filozoficzny} jest pierwszą częścią sześciotomowego cyklu opowiadającego o przygodach młodego czarodzieja Harry'ego Pottera. Autorką powieści jest \emph{J. K. Rowling}, angielska pisarka, nauczycielka literatury, która stworzyła jedną z najbardziej poczytnych i fascynujących książek dla młodzieży. Powieść porusza problem uniwersalny. Opowiada o tym, że nawet mała, na pierwszy rzut oka niepozorna istota może wpłynąć na losy świata i walczyć ze złem. Bohaterem jest Harry, syn sławnych czarodziejów - Potterów. Od urodzenia był naznaczony piętnem w postaci blizny na czole, która powstała w niebezpiecznych okolicznościach. Akcja powieści toczy się w Londynie pod koniec XX wieku oraz w szkole dla czarodziejów w \underline {Hogwarcie}.\par Jest wiele części Harry ego Pottera: Harry Potter i Kamień Filozoficzny Harry Potter i Komnata Tajemnic Harry Potter i Więzień Azkabanu Harry Potter i Czara Ognia Harry Potter i Zakon Feniksa Harry Potter i Książę Półkrwi Harry Potter i Insygnia Śmierci

\begin{table}[]
\begin{tabular}{|l|l|l|l|l|}
\hline
9-m & 2    & 2m  & 19   & 3    \\ \hline
-m  & 7-m  & 13  & 8    & 1999 \\ \hline
8   & -9   & 122 & 4m-9 & m    \\ \hline
12  & 6m+4 & 17  & 0    & 3    \\ \hline
\end{tabular}
\label{moje}
\end{table}

